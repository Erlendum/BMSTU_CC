\chapter*{ЗАКЛЮЧЕНИЕ}
\addcontentsline{toc}{chapter}{ЗАКЛЮЧЕНИЕ}

В ходе выполнения лабораторной работы была реализована программа
на языке Go, позволяющая распознавать цепочки регулярного языка.


В программе было реализовано построение НКА по произвольно
введённому регулярному выражению, по НКА был построен эквивалентный
ему ДКА, по ДКА был построен эквивалентный ему КА, имеющий
наименьшее возможное количество состояний (в соответствии с алгоритмом
Бржозовского), а также был смоделирован минимальный КА для входной
цепочки из терминалов исходной грамматики.


Таким образом, в результате выполнения лабораторной работы были
приобретены практические навыки реализации важнейших элементов
лексических анализаторов на примере распознавания цепочек регулярного
языка.
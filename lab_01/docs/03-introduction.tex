\chapter*{ОПИСАНИЕ ЗАДАНИЯ}
\addcontentsline{toc}{chapter}{ЗАДАНИЯ}

Цель работы: приобретение практических навыков реализации важнейших
элементов лексических анализаторов на примере распознавания цепочек
регулярного языка.

\subsubsection*{Задачи работы}
\begin{enumerate}[label=\arabic*)]
	\item ознакомиться с основными понятиями и определениями, лежащими в
	основе построения лексических анализаторов;
	\item прояснить связь между регулярным множеством, регулярным
	выражением, праволинейным языком, конечно-автоматным языком и
	недетерминированным конечно-автоматным языком;
	\item разработать, протестировать и отладить программу распознавания цепочек
	регулярного языка в соответствии с предложенным вариантом
	грамматики.
\end{enumerate}

\subsubsection*{Содержание работы (Вариант 4)}

Напишите программу, которая в качестве входа принимает произвольное
регулярное выражение и выполняет следующие преобразования.

\begin{enumerate}[label=\arabic*)]
	\item По регулярному выражению строит НКА.
	\item По НКА строит эквивалентный ему ДКА.
	\item По ДКА строит эквивалентный ему КА, имеющий наименьшее
	возможное количество состояний (алгоритм Бржозовского).
	\item Моделирует минимальный КА для входной цепочки из терминалов
	исходной грамматики.
\end{enumerate}
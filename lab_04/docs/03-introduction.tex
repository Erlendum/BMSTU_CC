\chapter{Теоретическая часть}

\textbf{Цель работы:} приобретение практических навыков реализации синтаксически управляемого перевода.

\textbf{Задачи работы:}

\begin{enumerate}
	\item Разработать, тестировать и отладить программу синтаксического анализа
	в соответствии с предложенным вариантом грамматики.
	\item Включить в программу синтаксического анализ семантические действия
	для реализации синтаксически управляемого перевода инфиксного
	выражения в обратную польскую нотацию.
\end{enumerate}

Грамматика по варианту:

\begin{framed}
	\ttfamily 
	\begin{alltt}
<выражение> ->
   <логическое выражение>
<логическое выражение> ->
    <логический одночлен> | 
    <логическое выражение> ! <логический одночлен>
<логический одночлен> ->
    <вторичное логическое выражение> | 
    <логический одночлен> \& <вторичное логическое выражение>
<вторичное логическое выражение> ->
    <первичное логическое выражение> |
    ~ <первичное логическое выражение>
<первичное логическое выражение> ->
    <логическое значение> |
<идентификатор>
    <логическое значение> ->
    true | false
<знак логической операции> ->
    ~ | & | !
\end{alltt}
\end{framed}

\chapter{Практическая часть}


\section{Результат выполнения работы}

В листинге~\ref{lst:input} представлены входные данные и выходные данные.

\begin{lstlisting}[language=Go, caption={Входная и выходные данные}, label=lst:input]
a & b ! ~c ! d & e ! f ! g ! h & k
RPN: [a b & c ~ ! d e & ! f ! g ! h k & !]
\end{lstlisting}

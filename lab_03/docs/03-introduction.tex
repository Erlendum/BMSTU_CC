\chapter{Теоретическая часть}

\textbf{Цель работы:} приобретение практических навыков реализации алгоритма рекурсивного спуска для разбора грамматики и построения синтаксического дерева.

\textbf{Задачи работы:}

\begin{enumerate}
	\item Познакомиться с методом рекурсивного спуска для синтаксического анализа.
	\item Разработать, тестировать и отладить программу построения синтаксического дерева методом рекурсивного спуска в соответствии с предложенным вариантом грамматики.
\end{enumerate}

\section{Задание}

\begin{enumerate}
	\item Дополнить грамматику по варианту блоком, состоящим из последовательности операторов присваивания (выбран стиль Си).
	\item Для модифицированной грамматики написать программу нисходящего синтаксического анализа с использованием метода рекурсивного спуска.
\end{enumerate}

Блок в стиле Си:

\begin{framed}
	\ttfamily 
	\begin{alltt}
<программа> -> 
    <блок>
<блок> -> 
    { <список операторов> }
<список операторов> ->
    <оператор> < хвост>
<хвост> ->
    ; <оператор> <хвост> | epsilon
\end{alltt}
\end{framed}

Грамматика по варианту:

\begin{framed}
	\ttfamily 
	\begin{alltt}
<выражение> ->
   <логическое выражение>
<логическое выражение> ->
    <логический одночлен> | 
    <логическое выражение> ! <логический одночлен>
<логический одночлен> ->
    <вторичное логическое выражение> | 
    <логический одночлен> \& <вторичное логическое выражение>
<вторичное логическое выражение> ->
    <первичное логическое выражение> |
    ~ <первичное логическое выражение>
<первичное логическое выражение> ->
    <логическое значение> |
<идентификатор>
    <логическое значение> ->
    true | false
<знак логической операции> ->
    ~ | & | !
\end{alltt}
\end{framed}

Грамматика по варианту после добавления блока:



\begin{framed}
	\ttfamily 
	\begin{alltt}
<программа> -> 
    <блок>
<блок> -> 
    { <список операторов> }
<список операторов> ->
    <оператор> < хвост>
<хвост> ->
    ; <оператор> <хвост> | epsilon


<выражение> ->
    <логическое выражение>
<логическое выражение> ->
    <логический одночлен> | 
    <логическое выражение> ! <логический одночлен>
<логический одночлен> ->
    <вторичное логическое выражение> | 
    <логический одночлен> \& <вторичное логическое выражение>
<вторичное логическое выражение> ->
    <первичное логическое выражение> |
    ~ <первичное логическое выражение>
<первичное логическое выражение> ->
    <логическое значение> |
    <идентификатор>
<логическое значение> ->
    true | false
<знак логической операции> ->
    ~ | & | !
\end{alltt}
\end{framed} 

\chapter{Практическая часть}


\section{Результат выполнения работы}

В листинге~\ref{lst:input} представлены входные данные. На рисунке~\ref{img:ast.png}~---~построенное AST-дерево.

\begin{lstlisting}[language=Go, caption={Входная программа}, label=lst:input]
{
	x = a & b !~c;
	y = d ! ~b & true;
	z = false
}
\end{lstlisting}

\includeimage
{ast}
{f} 
{H}
{0.9\textwidth}
{AST} 
